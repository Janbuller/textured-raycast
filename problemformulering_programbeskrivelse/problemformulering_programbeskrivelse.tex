\documentclass[11pt]{article}
\usepackage[utf8]{inputenc}
\usepackage[T1]{fontenc}
\usepackage{graphicx}
\usepackage{longtable}
\usepackage{wrapfig}
\usepackage{rotating}
\usepackage[normalem]{ulem}
\usepackage{amsmath}
\usepackage{amssymb}
\usepackage{capt-of}
\usepackage{hyperref}
\usepackage[danish]{babel}
\usepackage{parskip}
\usepackage{authblk}

\author{Lasse Lykke Jensen \and Benjamin Vesterager Bovbjerg}

\date{\today}
\title{Raycast Console RPG}
\hypersetup{
 pdfauthor={Lasse Lykke Jensen, Benjamin Vesterager Bovbjerg},
 pdftitle={Raycast Console RPG},
 pdfkeywords={},
 pdfsubject={},
 pdfcreator={},
 pdflang={Danish}}
\begin{document}

\maketitle

\section*{Problemformulering}
Vi syntes at konsoller i dag er meget undervurderet i forhold til underholdning,
multimedie og spil. Vi vil derfor lave et spil i en konsol, for at bevise
værdien. For at gøre dette, er det nødvendigt, at spillet er både sjovt, men
også teknologisk avanceret og imponerende. Vores problemstilling lyder derfor:

``Hvordan kan man lave et konsol-baseret spil, der er både sjovt, teknologisk
avanceret og teknologisk imponerende?''

\section*{Programbeskrivelse}
Vi vil lave en raycasting, 3D, openWorld, konsol-baseret RPG. Spillet skal være
indenfor genren ``Fantasy''. Vi har også tænkt os at lave et inventory system,
hvor man kan anskaffe og benytte forskellige ``items''. Vi vil også have
forskellige \textit{monstre} og lignende at bekæmpe. Til dette vil vi benytte et
turn-based fighting system. Menuen til systemet skal være 2D, men spilleren og
monstret skal vises i 3D.
\end{document}
